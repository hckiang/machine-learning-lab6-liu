%% -*- mode: latex; fill-column: 80; comment-column: 50; -*-
\documentclass[11pt,english]{article}

\usepackage[utf8]{inputenc}

\usepackage{pdfpages}
\usepackage{babel}
\usepackage{hyperref}
\usepackage{enumerate}
\usepackage{listings}
\usepackage{float}
\usepackage[justification=centering]{caption}
\usepackage{subcaption}
\usepackage{multirow}
\usepackage{diagbox}
\usepackage{amsmath}
\usepackage{amssymb}

\renewcommand*\ttdefault{lmtt}
\renewcommand*\sfdefault{lmss}

\DeclareMathOperator*{\argmax}{argmax}

\begin{document}

\author{\textbf{Group 2} \\* Arian Barakat \\* Hao Chi KIANG \\* Rebin Hosini \\ Yumeng Li}
\title{Introduction to Machine Learning: Lab 6}
\maketitle

\section*{Artificial Neural Network to Predict the
  Trigonometric Sine Function}



% \begin{figure}[H]
%   \centering
%   \includegraphics[width = 0.9\textwidth]{uniwinter.pdf}
%   \caption{Forecast result in winter}
%   \label{uniwinter}
% \end{figure}


\newpage
\lstset{
  frame=single,
  basicstyle=\ttfamily\footnotesize,
  commentstyle=\color{gray},
  frame=L,
  language=R,
  showstringspaces=false,
}

\newpage
\section*{Appendix 1: R code used}
\label{ax2}
\lstinputlisting{worksheet1.R}

\end{document}

