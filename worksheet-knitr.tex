\documentclass[11pt,english]{article}\usepackage[]{graphicx}\usepackage[]{color}
%% maxwidth is the original width if it is less than linewidth
%% otherwise use linewidth (to make sure the graphics do not exceed the margin)
\makeatletter
\def\maxwidth{ %
  \ifdim\Gin@nat@width>\linewidth
    \linewidth
  \else
    \Gin@nat@width
  \fi
}
\makeatother

\definecolor{fgcolor}{rgb}{0.345, 0.345, 0.345}
\newcommand{\hlnum}[1]{\textcolor[rgb]{0.686,0.059,0.569}{#1}}%
\newcommand{\hlstr}[1]{\textcolor[rgb]{0.192,0.494,0.8}{#1}}%
\newcommand{\hlcom}[1]{\textcolor[rgb]{0.678,0.584,0.686}{\textit{#1}}}%
\newcommand{\hlopt}[1]{\textcolor[rgb]{0,0,0}{#1}}%
\newcommand{\hlstd}[1]{\textcolor[rgb]{0.345,0.345,0.345}{#1}}%
\newcommand{\hlkwa}[1]{\textcolor[rgb]{0.161,0.373,0.58}{\textbf{#1}}}%
\newcommand{\hlkwb}[1]{\textcolor[rgb]{0.69,0.353,0.396}{#1}}%
\newcommand{\hlkwc}[1]{\textcolor[rgb]{0.333,0.667,0.333}{#1}}%
\newcommand{\hlkwd}[1]{\textcolor[rgb]{0.737,0.353,0.396}{\textbf{#1}}}%
\let\hlipl\hlkwb

\usepackage{framed}
\makeatletter
\newenvironment{kframe}{%
 \def\at@end@of@kframe{}%
 \ifinner\ifhmode%
  \def\at@end@of@kframe{\end{minipage}}%
  \begin{minipage}{\columnwidth}%
 \fi\fi%
 \def\FrameCommand##1{\hskip\@totalleftmargin \hskip-\fboxsep
 \colorbox{shadecolor}{##1}\hskip-\fboxsep
     % There is no \\@totalrightmargin, so:
     \hskip-\linewidth \hskip-\@totalleftmargin \hskip\columnwidth}%
 \MakeFramed {\advance\hsize-\width
   \@totalleftmargin\z@ \linewidth\hsize
   \@setminipage}}%
 {\par\unskip\endMakeFramed%
 \at@end@of@kframe}
\makeatother

\definecolor{shadecolor}{rgb}{.97, .97, .97}
\definecolor{messagecolor}{rgb}{0, 0, 0}
\definecolor{warningcolor}{rgb}{1, 0, 1}
\definecolor{errorcolor}{rgb}{1, 0, 0}
\newenvironment{knitrout}{}{} % an empty environment to be redefined in TeX

\usepackage{alltt}

\usepackage[utf8]{inputenc}

\usepackage{pdfpages}
\usepackage{babel}
\usepackage{hyperref}
\usepackage{enumerate}
\usepackage{listings}
\usepackage{float}
\usepackage[justification=centering]{caption}
\usepackage{subcaption}
\usepackage{multirow}
\usepackage{diagbox}
\usepackage{amsmath}
\usepackage{amssymb}

\renewcommand*\ttdefault{lmtt}
\renewcommand*\sfdefault{lmss}

\DeclareMathOperator*{\argmax}{argmax}
\IfFileExists{upquote.sty}{\usepackage{upquote}}{}
\begin{document}

\author{Hao Chi KIANG}
\title{Introduction to Machine Learning: Lab 3 Block 2}
\maketitle



\section*{Assignment 1: High-dimensional methods}
\subsection*{Assignment 1.1: Nearest Shrunken Centroid}



\autoref{cvmodel} shows the result of a cross validation. The plot shows
that some values between 1.0 and 1.5 or 2.5 to 3 would be good choice as it
shows the error rate are lowest in this region. Anything bigger than 2.5 is
having too few words and anything smaller than 1 seems to be over-fitting.
In our case, we would choose the model with threshold=1 in the
following discussion, because this value have more features, hence can
tolerate more diversity.

\begin{figure}[H]
  \centering
\begin{knitrout}\small
\definecolor{shadecolor}{rgb}{0.969, 0.969, 0.969}\color{fgcolor}
\includegraphics[width=\maxwidth]{figure/cvmodel-1} 

\end{knitrout}
\caption{Cross validation result with different threshold values}
\label{cvmodel}
\end{figure}

A centroid plot is shown in \autoref{centroid}. In this model, the chosen
words are listed as follows:

\begin{knitrout}\small
\definecolor{shadecolor}{rgb}{0.969, 0.969, 0.969}\color{fgcolor}\begin{kframe}
\begin{verbatim}
##       id           0-score   1-score  
##  [1,] "papers"     "-0.4229" "0.5787" 
##  [2,] "important"  "-0.3961" "0.542"  
##  [3,] "submission" "-0.3804" "0.5205" 
##  [4,] "position"   "0.3648"  "-0.4992"
##  [5,] "published"  "-0.3161" "0.4326" 
##  [6,] "call"       "-0.3133" "0.4288" 
##  [7,] "due"        "-0.3133" "0.4288" 
##  [8,] "conference" "-0.3124" "0.4275" 
##  [9,] "dates"      "-0.3124" "0.4275" 
## [10,] "candidates" "0.2958"  "-0.4048"
\end{verbatim}
\end{kframe}
\end{knitrout}

These words are intuitively good indication for conference-related mails:
seemingly most of these email would talk about submitting papers.

\begin{figure}[H]
  \centering
\begin{knitrout}\small
\definecolor{shadecolor}{rgb}{0.969, 0.969, 0.969}\color{fgcolor}
\includegraphics[width=\maxwidth]{figure/centroid-1} 

\end{knitrout}
\caption{Centroid plot of the model with threshold=1}
\label{centroid}
\end{figure}

Test error is 0.0526316, which is very low in this case.


\subsection*{Assignment 1.2: Elastic Net and Support Vector Machine}


\autoref{cvglm} shows the cross validation result of elastic net using
the $glmnet$ library. 0.104439 is the value of $\lambda$
selected by the library. The prediction error is 0.1052632,
significantly larger than that of nearest shrunken centroid method.

\begin{figure}[H]
  \centering
\begin{knitrout}\small
\definecolor{shadecolor}{rgb}{0.969, 0.969, 0.969}\color{fgcolor}
\includegraphics[width=\maxwidth]{figure/cvglm-1} 

\end{knitrout}
\caption{Cross validation result of Elastic net}
\label{cvglm}
\end{figure}

On the other hand, the SVM done with the following code gives the same error
rate than the nearest shrunken centroid:
\begin{knitrout}\small
\definecolor{shadecolor}{rgb}{0.969, 0.969, 0.969}\color{fgcolor}\begin{kframe}
\begin{alltt}
\hlkwd{library}\hlstd{(}\hlstr{'kernlab'}\hlstd{)}

\hlkwd{set.seed}\hlstd{(}\hlnum{12345}\hlstd{)}
\hlstd{svmmod} \hlkwb{<-} \hlkwd{ksvm}\hlstd{(Conference} \hlopt{~} \hlstd{.,}
               \hlkwc{data} \hlstd{= train,}
               \hlkwc{kernel} \hlstd{=} \hlstr{'vanilladot'}\hlstd{,}
               \hlkwc{scaled} \hlstd{= F)}
\hlstd{err.svmmod} \hlkwb{<-} \hlkwd{mean}\hlstd{(}\hlkwd{predict}\hlstd{(svmmod, test)} \hlopt{!=} \hlstd{test[,CONFERENCE])}
\hlkwd{print}\hlstd{(err.svmmod)}
\end{alltt}
\end{kframe}
\end{knitrout}

The following table summerises the error rates of the three methods:\\
\begin{tabular}{l|l}
Method & Error rate\\
\hline
Nearest Shrunken Centroid & 0.0526316\\
Elastic Net & 0.1052632\\
SVM & 0.0526316
\end{tabular}\\

I would prefer nearest shrunken centroid for its straight forward
interpretation.


\subsection*{Assignment 1.3: Benjamini-Hochberg Multiple Testing}



The following are the words whose hypothesis are rejected by my
implementation of the Benjamini-Hochberg, with $\alpha = 0.05$,
sorted in ascending order of their p-values.

\begin{knitrout}\small
\definecolor{shadecolor}{rgb}{0.969, 0.969, 0.969}\color{fgcolor}\begin{kframe}
\begin{verbatim}
##      [,1]        
## [1,] "important" 
## [2,] "position"  
## [3,] "papers"    
## [4,] "submission"
## [5,] "candidates"
## [6,] "published"
\end{verbatim}
\end{kframe}
\end{knitrout}

The top ten features are consistent with the previous results with different
methods. That these features has high p-values means that the we can decide
that the distribution of them are different between conference emails and
non-conference emails.


\section*{Assignment 2: Budget Online SVM}



\begin{figure}[H]
  \centering
\begin{knitrout}\small
\definecolor{shadecolor}{rgb}{0.969, 0.969, 0.969}\color{fgcolor}
\includegraphics[width=\maxwidth]{figure/spamplot-1} 

\end{knitrout}
\caption{Budget Online SVM Error Rates with $h = 1$.}
\label{spamplot}
\end{figure}


\section*{Appendix 1: All codes}
\begin{knitrout}\small
\definecolor{shadecolor}{rgb}{0.969, 0.969, 0.969}\color{fgcolor}\begin{kframe}
\begin{alltt}
\hlstd{opts_chunk}\hlopt{$}\hlkwd{set}\hlstd{(}\hlkwc{size} \hlstd{=} \hlstr{'small'}\hlstd{)}
\hlcom{######## Assignment 1.1}
\hlkwd{library}\hlstd{(}\hlstr{'pamr'}\hlstd{)}

\hlkwd{set.seed}\hlstd{(}\hlnum{12345}\hlstd{)}
\hlstd{dat0} \hlkwb{<-} \hlkwd{read.csv2}\hlstd{(}\hlstr{'../question/data.csv'}\hlstd{,}
                  \hlkwc{fileEncoding} \hlstd{=} \hlstr{'ISO-8859-1'}\hlstd{,}
                  \hlkwc{encoding} \hlstd{=} \hlstr{'UTF-8'}\hlstd{)}

\hlstd{dat} \hlkwb{<-} \hlkwd{as.data.frame}\hlstd{(dat0)}

\hlstd{CONFERENCE} \hlkwb{<-} \hlkwd{colnames}\hlstd{(dat)} \hlopt{==} \hlstr{'Conference'}
\hlstd{NON.CONFERENCE} \hlkwb{<-} \hlopt{!} \hlstd{CONFERENCE}

\hlstd{dat[ ,CONFERENCE]} \hlkwb{<-} \hlkwd{as.factor}\hlstd{(dat0[, CONFERENCE])}
\hlkwd{rownames}\hlstd{(dat)} \hlkwb{<-} \hlkwd{as.character}\hlstd{(}\hlnum{1}\hlopt{:}\hlkwd{nrow}\hlstd{(dat))}
\hlstd{train.idx} \hlkwb{<-} \hlkwd{sample}\hlstd{(}\hlkwd{nrow}\hlstd{(dat),} \hlkwd{round}\hlstd{(}\hlkwd{nrow}\hlstd{(dat)} \hlopt{*} \hlnum{0.7}\hlstd{))}
\hlstd{train} \hlkwb{<-} \hlstd{dat[train.idx, ]}
\hlstd{test} \hlkwb{<-} \hlstd{dat[}\hlopt{-}\hlstd{train.idx, ]}

\hlstd{totrain} \hlkwb{<-} \hlkwd{list}\hlstd{(}\hlkwc{x} \hlstd{=} \hlkwd{t}\hlstd{(train[,NON.CONFERENCE]),}
                \hlkwc{y} \hlstd{= train[,CONFERENCE],}
                \hlkwc{geneid} \hlstd{=} \hlkwd{colnames}\hlstd{(train),}
                \hlkwc{genenames} \hlstd{=} \hlkwd{colnames}\hlstd{(train))}


\hlstd{model} \hlkwb{<-} \hlkwd{pamr.train}\hlstd{(totrain,} \hlkwc{threshold} \hlstd{=} \hlkwd{seq}\hlstd{(}\hlnum{0}\hlstd{,}\hlnum{4}\hlstd{,} \hlnum{0.1}\hlstd{))}
\hlstd{cvmodel} \hlkwb{<-} \hlkwd{pamr.cv}\hlstd{(model, totrain)}
\hlstd{best} \hlkwb{<-} \hlnum{1.0}

\hlstd{best.words} \hlkwb{<-} \hlkwd{head}\hlstd{(} \hlkwd{pamr.listgenes}\hlstd{(model, totrain,}
                                   \hlkwc{threshold} \hlstd{= best),} \hlkwc{n} \hlstd{=} \hlnum{10}\hlstd{)}

\hlstd{test.error} \hlkwb{<-}
    \hlkwd{mean}\hlstd{(test[,CONFERENCE]} \hlopt{!=}
         \hlkwd{pamr.predict}\hlstd{(model,}
                      \hlkwd{t}\hlstd{(}\hlkwd{as.matrix}\hlstd{(test[,NON.CONFERENCE])),}
                      \hlstd{best))}
\hlkwd{pamr.plotcv}\hlstd{(cvmodel)}
\hlstd{best.words}
\hlkwd{pamr.plotcen}\hlstd{(model, totrain,} \hlkwc{threshold} \hlstd{= best)}
\hlkwd{library}\hlstd{(}\hlstr{'glmnet'}\hlstd{)}
\hlkwd{set.seed}\hlstd{(}\hlnum{12345}\hlstd{)}
\hlstd{glmmod} \hlkwb{<-} \hlkwd{cv.glmnet}\hlstd{(}\hlkwc{x} \hlstd{=} \hlkwd{as.matrix}\hlstd{(train[,NON.CONFERENCE]),}
                    \hlkwc{y} \hlstd{= train[,CONFERENCE],}
                    \hlkwc{alpha} \hlstd{=} \hlnum{0.5}\hlstd{,}
                    \hlkwc{family} \hlstd{=} \hlstr{'binomial'}\hlstd{)}

\hlstd{glmerr} \hlkwb{<-} \hlkwd{mean}\hlstd{((}\hlkwd{predict}\hlstd{(glmmod,}
                        \hlkwd{as.matrix}\hlstd{(test[,NON.CONFERENCE]),}
                        \hlkwc{s} \hlstd{=} \hlstr{"lambda.min"}\hlstd{)} \hlopt{>} \hlnum{0}\hlstd{)} \hlopt{+} \hlnum{1} \hlopt{==}
               \hlstd{test[,CONFERENCE])}
\hlcom{######## Assignment 1.2}
\hlkwd{plot}\hlstd{(glmmod)}
\hlkwd{library}\hlstd{(}\hlstr{'kernlab'}\hlstd{)}

\hlkwd{set.seed}\hlstd{(}\hlnum{12345}\hlstd{)}
\hlstd{svmmod} \hlkwb{<-} \hlkwd{ksvm}\hlstd{(Conference} \hlopt{~} \hlstd{.,}
               \hlkwc{data} \hlstd{= train,}
               \hlkwc{kernel} \hlstd{=} \hlstr{'vanilladot'}\hlstd{,}
               \hlkwc{scaled} \hlstd{= F)}
\hlstd{err.svmmod} \hlkwb{<-} \hlkwd{mean}\hlstd{(}\hlkwd{predict}\hlstd{(svmmod, test)} \hlopt{!=} \hlstd{test[,CONFERENCE])}
\hlkwd{print}\hlstd{(err.svmmod)}
\hlcom{####### Benjamini-Hochberg}
\hlcom{## Alpha value for Benjamini-Hochberg method}
\hlstd{bhalpha} \hlkwb{<-} \hlnum{0.05}

\hlstd{bh.featselect} \hlkwb{<-} \hlkwa{function} \hlstd{(}\hlkwc{x}\hlstd{,} \hlkwc{y}\hlstd{,} \hlkwc{alpha}\hlstd{) \{}
    \hlstd{sp} \hlkwb{<-} \hlkwd{split}\hlstd{(x, y)}

    \hlstd{pvals} \hlkwb{<-} \hlkwd{sapply}\hlstd{(}\hlnum{1}\hlopt{:}\hlkwd{ncol}\hlstd{(x),} \hlkwa{function} \hlstd{(}\hlkwc{feat}\hlstd{) \{}
        \hlkwd{t.test}\hlstd{(}\hlkwc{x} \hlstd{= sp[[}\hlnum{1}\hlstd{]][,feat],}
               \hlkwc{y} \hlstd{= sp[[}\hlnum{2}\hlstd{]][,feat],}
               \hlkwc{alternative} \hlstd{=} \hlstr{"two.sided"}\hlstd{)}\hlopt{$}\hlstd{p.value}
    \hlstd{\})} \hlcom{# Zero result means sample variance = 0: untestable}

    \hlkwd{names}\hlstd{(pvals)} \hlkwb{<-} \hlkwd{colnames}\hlstd{(x)}
    \hlstd{ord} \hlkwb{<-} \hlkwd{order}\hlstd{(pvals)}
    \hlstd{max.j} \hlkwb{<-} \hlkwd{max}\hlstd{(}\hlkwd{which}\hlstd{( pvals[ord]} \hlopt{<=}
                        \hlstd{(}\hlnum{1}\hlopt{:}\hlkwd{length}\hlstd{(pvals))} \hlopt{*}\hlstd{alpha} \hlopt{/}\hlkwd{length}\hlstd{(pvals)))}

    \hlstd{toreturn} \hlkwb{<-} \hlstd{ord[}\hlnum{1}\hlopt{:}\hlstd{max.j]}
    \hlkwd{names}\hlstd{(toreturn)} \hlkwb{<-} \hlkwd{colnames}\hlstd{(x)[toreturn]}
    \hlstd{toreturn}
\hlstd{\}}

\hlkwd{print}\hlstd{(}\hlkwd{t}\hlstd{(}\hlkwd{t}\hlstd{((}\hlkwd{names}\hlstd{(}
              \hlkwd{bh.featselect}\hlstd{(train[,NON.CONFERENCE],}
                            \hlstd{train[CONFERENCE], bhalpha))))))}

\hlkwd{set.seed}\hlstd{(}\hlnum{1234567890}\hlstd{)}
\hlstd{spam} \hlkwb{<-} \hlkwd{read.csv2}\hlstd{(}\hlstr{"../question/spambase.csv"}\hlstd{)}
\hlstd{ind} \hlkwb{<-} \hlkwd{sample}\hlstd{(}\hlnum{1}\hlopt{:}\hlkwd{nrow}\hlstd{(spam))}
\hlstd{spam} \hlkwb{<-} \hlstd{spam[ind,}\hlkwd{c}\hlstd{(}\hlnum{1}\hlopt{:}\hlnum{48}\hlstd{,}\hlnum{58}\hlstd{)]}
\hlstd{spam} \hlkwb{<-} \hlkwd{as.matrix}\hlstd{(spam)}          \hlcom{# Much, much faster computation}
\hlstd{h} \hlkwb{<-} \hlnum{1}
\hlstd{N} \hlkwb{<-} \hlnum{500}                         \hlcom{# number of training points}

\hlstd{SPAM.COL} \hlkwb{<-} \hlkwd{colnames}\hlstd{(spam)} \hlopt{==} \hlstr{'Spam'}
\hlstd{FEAT.COL} \hlkwb{<-} \hlopt{!} \hlstd{SPAM.COL}

\hlstd{gaussian_k} \hlkwb{<-} \hlkwa{function}\hlstd{(}\hlkwc{x}\hlstd{,} \hlkwc{h}\hlstd{) \{}          \hlcom{# Gaussian kernel}
    \hlkwd{exp}\hlstd{(}\hlopt{-}\hlstd{(x}\hlopt{^}\hlnum{2}\hlstd{)}\hlopt{/}\hlstd{(}\hlnum{2} \hlopt{*} \hlstd{h}\hlopt{^}\hlnum{2}\hlstd{))}
\hlstd{\}}

\hlstd{SVM} \hlkwb{<-} \hlkwd{Vectorize}\hlstd{(}\hlkwa{function}\hlstd{(}\hlkwc{sv}\hlstd{,} \hlkwc{i}\hlstd{) \{}
    \hlstd{dists} \hlkwb{<-} \hlkwd{sapply}\hlstd{(sv,} \hlkwa{function} \hlstd{(}\hlkwc{m}\hlstd{) \{}
        \hlkwd{gaussian_k}\hlstd{(}
            \hlkwd{dist}\hlstd{(}\hlkwd{matrix}\hlstd{(}\hlkwd{c}\hlstd{(spam[m,FEAT.COL], spam[i,FEAT.COL]),}
                        \hlkwc{nrow} \hlstd{=} \hlnum{2}\hlstd{,} \hlkwc{byrow} \hlstd{= T)), h)}
    \hlstd{\})}
    \hlkwd{t}\hlstd{(dists)} \hlopt \hlstd{(spam[sv,SPAM.COL]} \hlopt{*} \hlnum{2} \hlopt{-} \hlnum{1}\hlstd{)}
\hlstd{\},} \hlkwc{vectorize.args} \hlstd{=} \hlstr{'i'}\hlstd{)}

\hlstd{online} \hlkwb{<-} \hlkwa{function} \hlstd{(}\hlkwc{M}\hlstd{,} \hlkwc{beta}\hlstd{) \{}
    \hlstd{ptm} \hlkwb{<-} \hlkwd{proc.time}\hlstd{()}

    \hlstd{errors} \hlkwb{<-} \hlnum{1}
    \hlstd{errorrate} \hlkwb{<-} \hlkwd{vector}\hlstd{(}\hlkwc{length} \hlstd{= N)}
    \hlstd{errorrate[}\hlnum{1}\hlstd{]} \hlkwb{<-} \hlnum{1}
    \hlstd{sv} \hlkwb{<-} \hlkwd{c}\hlstd{(}\hlnum{1}\hlstd{)}
    \hlkwa{for}\hlstd{(i} \hlkwa{in} \hlnum{2}\hlopt{:}\hlstd{N) \{}
        \hlstd{yhat} \hlkwb{<-} \hlkwd{SVM}\hlstd{(sv, i)}
        \hlstd{yi} \hlkwb{<-} \hlstd{spam[i,SPAM.COL]} \hlopt{*} \hlnum{2} \hlopt{-} \hlnum{1}
        \hlstd{errors} \hlkwb{<-} \hlstd{errors} \hlopt{+} \hlstd{(yhat} \hlopt{*} \hlstd{yi} \hlopt{<} \hlnum{0}\hlstd{)}
        \hlstd{errorrate[i]} \hlkwb{<-} \hlstd{errors} \hlopt{/} \hlstd{i}

        \hlkwa{if} \hlstd{( yi} \hlopt{*} \hlstd{yhat} \hlopt{<=} \hlstd{beta ) \{}
            \hlstd{sv} \hlkwb{<-} \hlkwd{c}\hlstd{(sv, i)}

            \hlcom{## cat(length(sv), ' ')}
            \hlkwa{if} \hlstd{(}\hlkwd{length}\hlstd{(sv)} \hlopt{>} \hlstd{M) \{}
                \hlstd{svtags} \hlkwb{<-} \hlstd{spam[sv,SPAM.COL]} \hlopt{*}\hlnum{2} \hlopt{-}\hlnum{1}
                \hlstd{sv} \hlkwb{<-}
                    \hlstd{sv[}\hlopt{-}\hlkwd{which.max}\hlstd{(}
                            \hlstd{svtags} \hlopt{*} \hlstd{(}\hlkwd{SVM}\hlstd{(sv,sv)} \hlopt{-} \hlstd{svtags)}
                            \hlcom{## Because k(0,h) = 1}
                        \hlstd{)]}
            \hlstd{\}}
        \hlstd{\}}
        \hlcom{## cat('Step:', i,}
        \hlcom{##     'SV:', length(sv),}
        \hlcom{##     'Err:', errorrate[i],}
        \hlcom{##     'yhat:', yhat,}
        \hlcom{##     'product', (spam[i,'Spam'] * 2 - 1) * yhat, '\textbackslash{}n')}
    \hlstd{\}}

    \hlkwd{cat}\hlstd{(}\hlstr{'Terminated. M ='}\hlstd{, M,}\hlstr{'Beta ='}\hlstd{, beta,}
        \hlstr{'SV = '}\hlstd{,} \hlkwd{length}\hlstd{(sv),}
        \hlstr{'Elapsed ='}\hlstd{, (}\hlkwd{proc.time}\hlstd{()} \hlopt{-} \hlstd{ptm)[}\hlnum{3}\hlstd{],} \hlstr{'\textbackslash{}n'} \hlstd{)}
    \hlstd{errorrate}
\hlstd{\}}

\hlstd{errors} \hlkwb{<-} \hlkwd{matrix}\hlstd{(}
    \hlkwd{c}\hlstd{(}\hlkwd{online}\hlstd{(}\hlnum{500}\hlstd{,} \hlnum{0}\hlstd{),}
      \hlkwd{online}\hlstd{(}\hlnum{500}\hlstd{,} \hlopt{-}\hlnum{0.05}\hlstd{),}
      \hlkwd{online}\hlstd{(}\hlnum{20}\hlstd{,} \hlnum{0}\hlstd{),}
      \hlkwd{online}\hlstd{(}\hlnum{20}\hlstd{,} \hlopt{-}\hlnum{0.05}\hlstd{)),}
    \hlkwc{ncol} \hlstd{=} \hlnum{4}
\hlstd{)}


\hlkwd{matplot}\hlstd{(}
    \hlstd{errors,}
    \hlkwc{type} \hlstd{=} \hlkwd{c}\hlstd{(}\hlstr{"l"}\hlstd{),}
    \hlkwc{lty} \hlstd{=} \hlnum{1}\hlstd{,}
    \hlkwc{pch} \hlstd{=} \hlnum{1}\hlstd{,}
    \hlkwc{col} \hlstd{=} \hlnum{1}\hlopt{:}\hlkwd{ncol}\hlstd{(errors),}
    \hlkwc{main} \hlstd{=} \hlstr{'Error Rates of Different Parameters'}\hlstd{,}
    \hlkwc{xlab} \hlstd{=} \hlstr{'Sample Size'}\hlstd{,}
    \hlkwc{ylab} \hlstd{=} \hlstr{'Error Rate'}\hlstd{)}
\hlkwd{legend}\hlstd{(}\hlstr{"topright"}\hlstd{,} \hlkwc{legend} \hlstd{=} \hlkwd{c}\hlstd{(}\hlstr{'M = 500, Beta = 0'}\hlstd{,}
                              \hlstr{'M = 500, Beta = -0.05'}\hlstd{,}
                              \hlstr{'M = 20,  Beta = 0'}\hlstd{,}
                              \hlstr{'M = 20,  Beta = -0.05'}\hlstd{),}
       \hlkwc{col}\hlstd{=}\hlnum{1}\hlopt{:}\hlnum{4}\hlstd{,} \hlkwc{lty} \hlstd{=} \hlnum{1}\hlstd{)}
\end{alltt}
\end{kframe}
\end{knitrout}

\end{document}
